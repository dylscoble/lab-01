% Options for packages loaded elsewhere
\PassOptionsToPackage{unicode}{hyperref}
\PassOptionsToPackage{hyphens}{url}
%
\documentclass[
]{article}
\title{Dylan Scoble}
\author{STAT 108}
\date{1/11/2022}

\usepackage{amsmath,amssymb}
\usepackage{lmodern}
\usepackage{iftex}
\ifPDFTeX
  \usepackage[T1]{fontenc}
  \usepackage[utf8]{inputenc}
  \usepackage{textcomp} % provide euro and other symbols
\else % if luatex or xetex
  \usepackage{unicode-math}
  \defaultfontfeatures{Scale=MatchLowercase}
  \defaultfontfeatures[\rmfamily]{Ligatures=TeX,Scale=1}
\fi
% Use upquote if available, for straight quotes in verbatim environments
\IfFileExists{upquote.sty}{\usepackage{upquote}}{}
\IfFileExists{microtype.sty}{% use microtype if available
  \usepackage[]{microtype}
  \UseMicrotypeSet[protrusion]{basicmath} % disable protrusion for tt fonts
}{}
\makeatletter
\@ifundefined{KOMAClassName}{% if non-KOMA class
  \IfFileExists{parskip.sty}{%
    \usepackage{parskip}
  }{% else
    \setlength{\parindent}{0pt}
    \setlength{\parskip}{6pt plus 2pt minus 1pt}}
}{% if KOMA class
  \KOMAoptions{parskip=half}}
\makeatother
\usepackage{xcolor}
\IfFileExists{xurl.sty}{\usepackage{xurl}}{} % add URL line breaks if available
\IfFileExists{bookmark.sty}{\usepackage{bookmark}}{\usepackage{hyperref}}
\hypersetup{
  pdftitle={Dylan Scoble},
  pdfauthor={STAT 108},
  hidelinks,
  pdfcreator={LaTeX via pandoc}}
\urlstyle{same} % disable monospaced font for URLs
\usepackage[margin=1in]{geometry}
\usepackage{color}
\usepackage{fancyvrb}
\newcommand{\VerbBar}{|}
\newcommand{\VERB}{\Verb[commandchars=\\\{\}]}
\DefineVerbatimEnvironment{Highlighting}{Verbatim}{commandchars=\\\{\}}
% Add ',fontsize=\small' for more characters per line
\usepackage{framed}
\definecolor{shadecolor}{RGB}{248,248,248}
\newenvironment{Shaded}{\begin{snugshade}}{\end{snugshade}}
\newcommand{\AlertTok}[1]{\textcolor[rgb]{0.94,0.16,0.16}{#1}}
\newcommand{\AnnotationTok}[1]{\textcolor[rgb]{0.56,0.35,0.01}{\textbf{\textit{#1}}}}
\newcommand{\AttributeTok}[1]{\textcolor[rgb]{0.77,0.63,0.00}{#1}}
\newcommand{\BaseNTok}[1]{\textcolor[rgb]{0.00,0.00,0.81}{#1}}
\newcommand{\BuiltInTok}[1]{#1}
\newcommand{\CharTok}[1]{\textcolor[rgb]{0.31,0.60,0.02}{#1}}
\newcommand{\CommentTok}[1]{\textcolor[rgb]{0.56,0.35,0.01}{\textit{#1}}}
\newcommand{\CommentVarTok}[1]{\textcolor[rgb]{0.56,0.35,0.01}{\textbf{\textit{#1}}}}
\newcommand{\ConstantTok}[1]{\textcolor[rgb]{0.00,0.00,0.00}{#1}}
\newcommand{\ControlFlowTok}[1]{\textcolor[rgb]{0.13,0.29,0.53}{\textbf{#1}}}
\newcommand{\DataTypeTok}[1]{\textcolor[rgb]{0.13,0.29,0.53}{#1}}
\newcommand{\DecValTok}[1]{\textcolor[rgb]{0.00,0.00,0.81}{#1}}
\newcommand{\DocumentationTok}[1]{\textcolor[rgb]{0.56,0.35,0.01}{\textbf{\textit{#1}}}}
\newcommand{\ErrorTok}[1]{\textcolor[rgb]{0.64,0.00,0.00}{\textbf{#1}}}
\newcommand{\ExtensionTok}[1]{#1}
\newcommand{\FloatTok}[1]{\textcolor[rgb]{0.00,0.00,0.81}{#1}}
\newcommand{\FunctionTok}[1]{\textcolor[rgb]{0.00,0.00,0.00}{#1}}
\newcommand{\ImportTok}[1]{#1}
\newcommand{\InformationTok}[1]{\textcolor[rgb]{0.56,0.35,0.01}{\textbf{\textit{#1}}}}
\newcommand{\KeywordTok}[1]{\textcolor[rgb]{0.13,0.29,0.53}{\textbf{#1}}}
\newcommand{\NormalTok}[1]{#1}
\newcommand{\OperatorTok}[1]{\textcolor[rgb]{0.81,0.36,0.00}{\textbf{#1}}}
\newcommand{\OtherTok}[1]{\textcolor[rgb]{0.56,0.35,0.01}{#1}}
\newcommand{\PreprocessorTok}[1]{\textcolor[rgb]{0.56,0.35,0.01}{\textit{#1}}}
\newcommand{\RegionMarkerTok}[1]{#1}
\newcommand{\SpecialCharTok}[1]{\textcolor[rgb]{0.00,0.00,0.00}{#1}}
\newcommand{\SpecialStringTok}[1]{\textcolor[rgb]{0.31,0.60,0.02}{#1}}
\newcommand{\StringTok}[1]{\textcolor[rgb]{0.31,0.60,0.02}{#1}}
\newcommand{\VariableTok}[1]{\textcolor[rgb]{0.00,0.00,0.00}{#1}}
\newcommand{\VerbatimStringTok}[1]{\textcolor[rgb]{0.31,0.60,0.02}{#1}}
\newcommand{\WarningTok}[1]{\textcolor[rgb]{0.56,0.35,0.01}{\textbf{\textit{#1}}}}
\usepackage{longtable,booktabs,array}
\usepackage{calc} % for calculating minipage widths
% Correct order of tables after \paragraph or \subparagraph
\usepackage{etoolbox}
\makeatletter
\patchcmd\longtable{\par}{\if@noskipsec\mbox{}\fi\par}{}{}
\makeatother
% Allow footnotes in longtable head/foot
\IfFileExists{footnotehyper.sty}{\usepackage{footnotehyper}}{\usepackage{footnote}}
\makesavenoteenv{longtable}
\usepackage{graphicx}
\makeatletter
\def\maxwidth{\ifdim\Gin@nat@width>\linewidth\linewidth\else\Gin@nat@width\fi}
\def\maxheight{\ifdim\Gin@nat@height>\textheight\textheight\else\Gin@nat@height\fi}
\makeatother
% Scale images if necessary, so that they will not overflow the page
% margins by default, and it is still possible to overwrite the defaults
% using explicit options in \includegraphics[width, height, ...]{}
\setkeys{Gin}{width=\maxwidth,height=\maxheight,keepaspectratio}
% Set default figure placement to htbp
\makeatletter
\def\fps@figure{htbp}
\makeatother
\setlength{\emergencystretch}{3em} % prevent overfull lines
\providecommand{\tightlist}{%
  \setlength{\itemsep}{0pt}\setlength{\parskip}{0pt}}
\setcounter{secnumdepth}{-\maxdimen} % remove section numbering
\ifLuaTeX
  \usepackage{selnolig}  % disable illegal ligatures
\fi

\begin{document}
\maketitle

\hypertarget{data-trails-in-san-francisco-ca.}{%
\section{Data: Trails in San Francisco,
CA.}\label{data-trails-in-san-francisco-ca.}}

Today's data comes from the
\href{https://opendata.mtc.ca.gov/datasets/MTC::san-francisco-bay-trail/about}{Metropolitan
Transportation Commission (MTC) Open Data Catalog} an Open Data program
managed by the MTC and the Association of Bay Area Governments to
provide local agencies and the public with their data needs.

In this lab, we will focus on data about the existing and planned
segments of the San Francisco Bay trail. The data is located in the
\emph{SFO\_trails.csv} file located in the \emph{data} folder. Use the
code below to read in the .csv file and save it in the RStudio
environment as a data frame called \texttt{trails}.

\begin{Shaded}
\begin{Highlighting}[]
\NormalTok{trails }\OtherTok{\textless{}{-}} \FunctionTok{read\_csv}\NormalTok{(}\StringTok{"data/SFO{-}trails.csv"}\NormalTok{)}
\end{Highlighting}
\end{Shaded}

A full list of the variables in the dataset is available
\href{https://opendata.mtc.ca.gov/datasets/MTC::san-francisco-bay-trail/about}{here}.
For today's analysis, we will primarily focus on the following
variables:

\begin{longtable}[]{@{}ll@{}}
\toprule
\endhead
\texttt{status} & Whether the trail is proposed or existing \\
\texttt{class} & Category for the trail segment (4 types) \\
\texttt{length} & Length of the trail segment in miles \\
\bottomrule
\end{longtable}

\hypertarget{exercises}{%
\section{Exercises}\label{exercises}}

\textbf{Write your answers in complete sentences and show all code and
output.}

Before doing any analysis, we may want to get quick view of the data.
This is a useful thing to do after importing data to see if the data
imported correctly. One way to do this, is to look at the actual
dataset. Type the code below in the \textbf{console} to view the entire
dataset.

\begin{Shaded}
\begin{Highlighting}[]
\FunctionTok{View}\NormalTok{(trails)}
\end{Highlighting}
\end{Shaded}

\hypertarget{exploratory-data-analysis}{%
\subsection{Exploratory Data Analysis}\label{exploratory-data-analysis}}

\begin{enumerate}
\def\labelenumi{\arabic{enumi}.}
\item
  Now that we've had a quick view of the dataset, let's get more details
  about its structure. Sometimes viewing a summary of the data structure
  is more useful than viewing the raw data, especially if the dataset
  has a large number of observations and/or rows. Run the code below to
  use the \texttt{glimpse} function to see a summary of the
  \texttt{trails} dataset.

  How many observations are in the \texttt{trails} dataset? How many
  variables? There are 739 total observations in the `trails' dataset.
  In each observation, there are 12 variables.
\end{enumerate}

\begin{Shaded}
\begin{Highlighting}[]
\FunctionTok{glimpse}\NormalTok{(trails)}
\end{Highlighting}
\end{Shaded}

\begin{enumerate}
\def\labelenumi{\arabic{enumi}.}
\setcounter{enumi}{1}
\item
  Before conducting statistical inference (or eventually fitting
  regression models), we need do some exploratory data analysis (EDA).
  Much of EDA consists of visualizing the data but it also includes
  calculating summary statistics for the variables in our dataset. Let's
  begin by examining the distribution of \texttt{status} with a data
  visualization and summary statistics.

  \begin{itemize}
  \item
    What is a type of graph that's appropriate to visualize the
    distribution of \texttt{status}? Fill in the \texttt{ggplot} code
    below to plot the distribution of \texttt{status}. Include
    informative axis labels and title on the graph.
  \item
    Then, calculate the proportion of observations in each category of
    \texttt{status} by completing the code below.
  \end{itemize}
\end{enumerate}

\begin{Shaded}
\begin{Highlighting}[]
\FunctionTok{ggplot}\NormalTok{(}\AttributeTok{data =}\NormalTok{ trails, }\FunctionTok{aes}\NormalTok{(}\AttributeTok{x =}\NormalTok{ status)) }\SpecialCharTok{+} 
  \FunctionTok{geom\_bar}\NormalTok{() }\SpecialCharTok{+} 
  \FunctionTok{labs}\NormalTok{(}\AttributeTok{x =} \StringTok{"Status"}\NormalTok{, }
       \AttributeTok{y =} \StringTok{"\# of Observations"}\NormalTok{, }
       \AttributeTok{title =} \StringTok{"San Francisco Trails"}\NormalTok{)}
\end{Highlighting}
\end{Shaded}

\begin{Shaded}
\begin{Highlighting}[]
\NormalTok{trails }\SpecialCharTok{\%\textgreater{}\%}
  \FunctionTok{count}\NormalTok{(status) }\SpecialCharTok{\%\textgreater{}\%}
  \FunctionTok{mutate}\NormalTok{(}\AttributeTok{proportion =}\NormalTok{ n }\SpecialCharTok{/} \FunctionTok{sum}\NormalTok{(n))}
\end{Highlighting}
\end{Shaded}

\begin{enumerate}
\def\labelenumi{\arabic{enumi}.}
\setcounter{enumi}{2}
\tightlist
\item
  Since we want to analyze characteristics for trails in the Bay Area,
  we will just use data from currently existing trails for the remainder
  of the analysis. Complete the code below to use the \texttt{filter}
  function to create a subset consisting only of trails that currently
  exist and have a value reported for \texttt{length}. Assign the subset
  the name \texttt{current\_trails}. (\emph{Hint: There should be 493
  observations in current\_trails.})
\end{enumerate}

\begin{Shaded}
\begin{Highlighting}[]
\NormalTok{current\_trails }\OtherTok{\textless{}{-}}\NormalTok{ trails }\SpecialCharTok{\%\textgreater{}\%}
  \FunctionTok{filter}\NormalTok{(status }\SpecialCharTok{==} \StringTok{"Existing"}\NormalTok{, }\SpecialCharTok{!}\FunctionTok{is.na}\NormalTok{(trails))}
\FunctionTok{glimpse}\NormalTok{(current\_trails)}
\end{Highlighting}
\end{Shaded}

\emph{This is a good place to knit, commit, and push changes to your
remote lab-01 repo on GitHub. Be sure to write an informative commit
message (e.g.~``Completed exercises 1 - 3''), and push every file to
GitHub by clicking the checkbox next to each file in the Git pane. After
you push the changes, the Git pane in RStudio should be empty.''}

\textbf{Use \texttt{current\_trails} for Exercises 4 - 7.}

\begin{enumerate}
\def\labelenumi{\arabic{enumi}.}
\setcounter{enumi}{3}
\item
  Let's examine the distribution of \texttt{length}. One important part
  of EDA is creating data visualizations to see the shape, center,
  spread, and outliers in a distribution. Data visualizations are also
  useful for examining the relationship between multiple variables.
  There are a lot of ways to make data visualizations in R; we will use
  the functions available in the \texttt{ggplot2} package.

  Make a graph to visualize the distribution of \texttt{length}. Include
  an informative title and axis labels.
\end{enumerate}

\begin{Shaded}
\begin{Highlighting}[]
\FunctionTok{ggplot}\NormalTok{(}\AttributeTok{data =}\NormalTok{ current\_trails) }\SpecialCharTok{+}
  \FunctionTok{geom\_histogram}\NormalTok{(}\AttributeTok{mapping =} \FunctionTok{aes}\NormalTok{(}\AttributeTok{x =}\NormalTok{ \_\_\_\_\_), }\AttributeTok{binwidth =} \FloatTok{0.5}\NormalTok{)}

\FunctionTok{ggplot}\NormalTok{(}\AttributeTok{data =}\NormalTok{ current\_trails, }\AttributeTok{mapping =} \FunctionTok{aes}\NormalTok{(}\AttributeTok{x =} \FunctionTok{as.factor}\NormalTok{(class), }\AttributeTok{y =}\NormalTok{ \_\_\_\_\_)) }\SpecialCharTok{+}
  \FunctionTok{geom\_boxplot}\NormalTok{()}
\end{Highlighting}
\end{Shaded}

See
\href{https://r4ds.had.co.nz/exploratory-data-analysis.html\#introduction-3}{Section
7.3.1 ``Visualizing Distributions''} or the
\href{https://ggplot2.tidyverse.org/reference/index.html}{ggplot2
reference page} for details and example code.

\begin{enumerate}
\def\labelenumi{\arabic{enumi}.}
\setcounter{enumi}{4}
\tightlist
\item
  Next, fill in the code below to use the \texttt{summarise} function to
  calculate various summary statistics for the variable \texttt{length}.
  You can use the
  \href{https://dplyr.tidyverse.org/reference/summarise.html}{summarise
  reference page} for more information about the function and example
  code.
\end{enumerate}

\begin{Shaded}
\begin{Highlighting}[]
\NormalTok{current\_trails }\SpecialCharTok{\%\textgreater{}\%}
  \FunctionTok{summarise}\NormalTok{(}\AttributeTok{min =}\NormalTok{ \_\_\_\_\_\_\_\_\_\_, }
            \AttributeTok{q1 =}\NormalTok{ \_\_\_\_\_\_\_\_\_\_, }
            \AttributeTok{median =}\NormalTok{ \_\_\_\_\_\_\_\_\_\_, }
            \AttributeTok{q3 =}\NormalTok{ \_\_\_\_\_\_\_\_\_\_, }
            \AttributeTok{max =}\NormalTok{ \_\_\_\_\_\_\_\_\_\_, }
            \AttributeTok{iqr =}\NormalTok{ \_\_\_\_\_\_\_\_\_\_, }
            \AttributeTok{mean =}\NormalTok{ \_\_\_\_\_\_\_\_\_\_, }
            \AttributeTok{std\_dev =}\NormalTok{ \_\_\_\_\_\_\_\_\_\_}
\NormalTok{            )}
\end{Highlighting}
\end{Shaded}

\begin{enumerate}
\def\labelenumi{\arabic{enumi}.}
\setcounter{enumi}{5}
\item
  Describe the distribution of \texttt{length}. Your description should
  include comments about the shape, center, spread, and any potential
  outliers. Use the graph from Exercise 4 and relevant summary
  statistics from Exercise 5 in your description.
\item
  We want to limit the analysis to trails that are more likely intended
  for day hikes, rather than multi-day hikes and camping. Therefore,
  let's remove the extreme outliers from the data for this analysis and
  only consider those trails that are 5 miles or shorter.

  Filter the dataset to remove the extreme outliers. \textbf{Be sure to
  save the updated dataset, so you can use it for the remainder of the
  lab.}
\end{enumerate}

\emph{This is a good place to knit, commit, and push changes to your
remote lab-01 repo on GitHub. Be sure to write informative commit
message (e.g.~``Completed exercises 4 - 7''), and push every file to
GitHub by clicking the checkbox next to each file in the Git pane. After
you push the changes, the Git pane in RStudio should be empty.''}

\begin{enumerate}
\def\labelenumi{\arabic{enumi}.}
\setcounter{enumi}{7}
\tightlist
\item
  Consider the distribution of \texttt{class}.

  \begin{itemize}
  \tightlist
  \item
    What are the values of \texttt{class} in the dataset? Show the code
    and output to support your answer.
  \item
    What do you think is the most likely reason for the missing
    observations of \texttt{class}? In other words, what does a missing
    value of \texttt{class} indicate?
  \end{itemize}
\item
  Complete the code below to impute (i.e.~fill in) the missing values of
  \texttt{class} with the appropriate value. After that, eliminate all
  the observations from class = 3, since we are not going to use the.
  Then, display the distribution of \texttt{class} to check that the
  missing values were correctly imputed.
\end{enumerate}

\begin{Shaded}
\begin{Highlighting}[]
\NormalTok{\_\_\_\_\_\_\_\_\_\_ }\OtherTok{\textless{}{-}}\NormalTok{ \_\_\_\_\_\_\_\_\_\_ }\SpecialCharTok{\%\textgreater{}\%} \FunctionTok{filter}\NormalTok{(class }\SpecialCharTok{!=}\DecValTok{3}\NormalTok{) }\SpecialCharTok{\%\textgreater{}\%} 
  \FunctionTok{mutate}\NormalTok{(}\AttributeTok{class =} \FunctionTok{if\_else}\NormalTok{(}\FunctionTok{is.na}\NormalTok{(class),\_\_\_\_\_\_,class)) }
\end{Highlighting}
\end{Shaded}

\begin{enumerate}
\def\labelenumi{\arabic{enumi}.}
\setcounter{enumi}{9}
\item
  Now that we've completed the univariate EDA (i.e.~examining one
  variable at a time), let's examine the relationship between the length
  of the trail and its class variable. Make a graph to visualize the
  relationship between \texttt{length} and \texttt{class} and calculate
  the appropriate summary statistics. Include informative axis labels
  and title on your graph.
\item
  Describe the relationship between \texttt{length} and \texttt{class}.
  In other words, describe how the distribution of \texttt{length}
  compares between trails that have different classes (1 = shared use
  bicycle and pedestrian path, 2 = bike lane, and 3 = bike route).
  Include information from the graph and summary statistics from the
  previous exercise in your response.
\end{enumerate}

\emph{This is a good place to knit, commit, and push changes to your
remote lab-01 repo on GitHub. Be sure to write informative commit
message (e.g.~``Completed exercises 8 - 11''), and push every file to
GitHub by clicking the checkbox next to each file in the Git pane. After
you push the changes, the Git pane in RStudio should be empty.''}

\hypertarget{statistical-inference}{%
\subsection{Statistical Inference}\label{statistical-inference}}

We'd like to use the data from the trails in SFO to make more general
conclusions about trails in urban areas in California, United States. We
will reasonably consider the trails in SFO representative of the trails
in other urban areas in the West Coast of United States.

Over the next few questions, will use statistical inference to assess
whether there is a difference in the mean length of trails that share
use bicycle and pedestrian path (class = 1) and those that only have a
bike line (class = 2).

\begin{enumerate}
\def\labelenumi{\arabic{enumi}.}
\setcounter{enumi}{11}
\tightlist
\item
  The following conditions must be met when we conduct statistical
  inference on the difference in means between two groups. For each
  condition, specify whether it is met and a brief explanation of your
  reasoning.

  \begin{itemize}
  \tightlist
  \item
    \textbf{Independence}
  \item
    \textbf{Sample Size}
  \item
    \textbf{Independent Groups}
  \end{itemize}
\item
  While we have observed a small difference in the mean length in trails
  with bike lanes (class = 2) and trials that share bikes with
  pedestrians (class = 1), let's assess if there is enough evidence to
  consider the difference ``statistically significant'' or if it appears
  to be due to random chance.
\end{enumerate}

The null and alternative hypotheses are written in statistical notation
below. State the hypotheses in words in the context of this analysis.

\[H_0: \mu_1 - \mu_2 = 0 \\
H_a: \mu_1 - \mu_2 \neq 0\]

\begin{enumerate}
\def\labelenumi{\arabic{enumi}.}
\setcounter{enumi}{13}
\tightlist
\item
  Fill in the code below to use the \texttt{t.test} function to
  calculate the test statistic and p-value. Replace \texttt{response}
  with the variable we're interested in drawing conclusions about and
  \texttt{group\_var} with the variable used to define the two groups.
\end{enumerate}

\begin{Shaded}
\begin{Highlighting}[]
\NormalTok{?t.test }\CommentTok{\# to see the help page from the function}
\FunctionTok{t.test}\NormalTok{(response }\SpecialCharTok{\textasciitilde{}}\NormalTok{ group\_var, }\AttributeTok{data =}\NormalTok{ \_\_\_\_\_\_\_\_, }
       \AttributeTok{alternative =} \StringTok{"\_\_\_\_\_\_\_\_\_\_\_"}\NormalTok{, }
       \AttributeTok{conf.level =} \FloatTok{0.99}\NormalTok{) }\CommentTok{\#less, greater, or two.sided}
\end{Highlighting}
\end{Shaded}

\begin{enumerate}
\def\labelenumi{\arabic{enumi}.}
\setcounter{enumi}{14}
\tightlist
\item
  Use the output from the previous exercise to answer the following:

  \begin{itemize}
  \tightlist
  \item
    Write the definition of the test statistic in the context of this
    analysis.
  \item
    Write the definition of the p-value in the context of this analysis.
  \item
    State your conclusion in the context of this analysis. Use a
    significance level of \(\alpha = 0.01\).
  \end{itemize}
\item
  Notice the confidence interval for the difference in mean trail length
  printed in the output from Exercise 14. Interpret this confidence
  interval in the context of this analysis.
\end{enumerate}

\emph{You're done and ready to submit your work! Knit, commit, and push
all remaining changes. You can use the commit message ``Done with Lab
1!'', and make sure you have pushed all the files to GitHub (your Git
pane in RStudio should be empty) and thatall documents are updated in
your repo on GitHub. Then submit the pdf for your assignment on
Gradescope. Include your repo name, so I can check your commits.}

\end{document}
